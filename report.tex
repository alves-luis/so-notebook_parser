\documentclass[a4paper]{article}

\usepackage[utf8]{inputenc}
\usepackage[portuges]{babel}
\usepackage{graphicx}
\usepackage[colorlinks = true, urlcolor = blue, linkcolor = black]{hyperref}
\usepackage{tabto}

\begin{document}

\title{Relatório do Trabalho Prático de POO}
\author{
José Costa (A82136)\\\\
\and Luís Alves (A80165)\\\\
\and Miguel Carvalho (A81909)\\\\
\textbf{Grupo 52}
}
\date{\today}

\maketitle
\tableofcontents

\section{Introdução}

	\tab Este projeto foi desenvolvido no âmbito da Unidade Curricular de
	\textit{Sistemas Operativos}, sendo que foi proposto como trabalho
	prático, a criação de um processador de notebooks de comandos.

\section{Sobre o programa}

	\subsection{Parsing}
		\tab Ao realizarmos parsing de um notebook, recolhemos a seguinte informação
		útil sobre cada comando:
		\begin{itemize}
			\item Comando com os respetivos argumentos
			\item Index (n) do comando no notebook
			\item Index do comando cujo output é preciso para executar este comando
			\item Nº de comandos que necessitam do output deste comando
			\item Array com os indexes dos comandos que necessitam do output deste comando
		\end{itemize}
		\par No entanto, guardamos o comando independentemente dos argumentos, que apenas
		são parsed devidamente aquando da execução do comando (através da função \texttt{get\_args()}).

	\subsection{Execução de Comandos}
		\tab Após realizarmos o parsing dos comandos e termos para cada um as informações
		supra mencionadas, corremos um ciclo de execução para cada um, podendo eles
		correr concorrentemente.
		\par Criamos então um filho, que, caso não requeira o output de nenhum comando,
		executa regularmente tendo o output redirecionado para um pipe de output. Caso
		requeira o output de outro comando, duplica um pipe para o input do comando a executar,
		\par Após a execução, o pai lê o output do filho e coloca esse output noutro conjunto
		de pipes, um para cada comando que vá precisar deste output como input, e um outro
		que servirá para colocar este output no notebook final.

	\subsection{Pipes e Pastas}
		\tab Para guardarmos os resultados intermédios (ver acima), necessitamos de guardar
		tanto os pipes de input, como os de output. Para isso, criamos uma pasta na diretoria
		\texttt{/tmp/SO}, sendo que os Pipes de Input têm o seguinte formato: /tmp/SO\_f\_t, sendo
		f o index do comando que fornece o output para o comando de index t.
		\par Já os Pipes de output simples, têm o formato /tmp/SO\_t, sendo t o index do output
		do n comando.
		\par De forma a não alterarmos o ficheiro em caso de término precoce (usando CTRL+C),
		criamos apenas um ficheiro temporário final no fim do processamento de todos os comandos,
		sendo que apenas terminado esse processo, esse ficheiro é copiado para o ficheiro
		original.

	\subsection{Limitações conhecidas}
		\tab Após vários testes, concluímos que o nosso programa tem algumas limitações,
		que não conseguimos remover a tempo da entrega:
			\begin{itemize}
				\item Outputs de tamanho superior a 65k (tamanho de um pipe com nome)
				\item Comandos que contenham pipes (comando1 | comando2)
				\item Por vezes não é possível eliminar as pastas criadas no /tmp
				\item Comandos que não terminem não forçam o término do processamento
				\item Outros erros esporádicos
			\end{itemize}
		\par Com os outros erros esporádicos, queremos indicar que por vezes, há notebooks
		que correram bem ao testar num PC a correr o subsystem de linux no Windows, mas
		não correram bem (algumas vezes) num PC a correr Ubuntu nativo. Dada a natureza
		esporádica desses erros e a proximidade da data de entrega aquando da sua descoberta,
		fica aqui documentada a sua existência.

	\section{Conclusão}
	\tab Com a realização deste trabalho, foi-nos possível aplicar as técnicas
	de programação em Unix lecionadas nas aulas num projeto mais concreto e maior
	do que os sugeridos nos exercícios. Para além disso, contribuiu para uma maior
	preparação para o teste.
	\par No entanto, a aparente simplicidade do projeto depressa se mostrou isso mesmo,
	apenas uma aparência, já que de forma a poder correr tudo conforme os requisitos
	e sem limitações, exigiria uma maior carga de trabalho que não pudemos disponibilizar.



\end{document}
